% In this file you should put all LaTeX macros and settings to be used both by
% the pdf version and the web version.
% This should be most of your macros.

% The theorem-like environments defined below are those that appear by default
% in the dependency graph. See the README of leanblueprint if you need help to
% customize this.
% The configuration below use the theorem counter for all those environments
% (this is what the [theorem] arguments mean) and never resets it.
% If you want for instance to number them within chapters then you can add
% [chapter] at the end of the next line.
\newtheorem{theorem}{Theorem}
\newtheorem{proposition}[theorem]{Proposition}
\newtheorem{lemma}[theorem]{Lemma}
\newtheorem{corollary}[theorem]{Corollary}

\theoremstyle{definition}
\newtheorem{definition}[theorem]{Definition}



\newcommand{\Z}{\mathbb{Z}}
\newcommand{\N}{\mathbb{N}}
\newcommand{\A}{\mathbb{A}}
\newcommand{\Q}{\mathbb{Q}}
\newcommand{\R}{\mathbb{R}}
\newcommand{\F}{\mathbb{F}}
\newcommand{\Qp}{\mathbb{Q}_p}
\newcommand{\Ql}{\mathbb{Q}_\ell}
\newcommand{\Qbar}{\overline{\Q}}
\newcommand{\Qpbar}{\overline{\Q}_p}
\newcommand{\Qlbar}{\overline{\Q}_\ell}
\newcommand{\bbC}{\mathbb{C}}
\newcommand{\GQ}{\Gal(\Qbar/\Q)}
\newcommand{\GQp}{\Gal(\Qpbar/\Qp)}
\newcommand{\GQl}{\Gal(\Qlbar/\Ql)}
\newcommand{\m}{\mathfrak{m}}
\newcommand{\GK}{\Gal(K^{\sep}/K)}
\newcommand{\GN}{\Gal(\overline{N}/N)}
\newcommand{\Kbar}{\overline{K}}
\newcommand{\Qhat}{\widehat{\Q}}
\newcommand{\calO}{\mathcal{O}}
\newcommand{\calOhat}{\widehat{\calO}}
\newcommand{\bbH}{\mathbb{H}}
\newcommand{\p}{{\mathfrak{p}}}
\newcommand{\rhobar}{\overline{\rho}}
\newcommand{\Zhat}{\widehat{\Z}}
\DeclareMathOperator{\Gal}{Gal}
\DeclareMathOperator{\avoid}{avoid}
\DeclareMathOperator{\Aut}{Aut}
\DeclareMathOperator{\GL}{GL}
\DeclareMathOperator{\PGL}{PGL}
\DeclareMathOperator{\PSL}{PSL}
\DeclareMathOperator{\SL}{SL}
\DeclareMathOperator{\Spec}{Spec}
\DeclareMathOperator{\sep}{sep}
\DeclareMathOperator{\ab}{ab}
\DeclareMathOperator{\tr}{tr}
\DeclareMathOperator{\Hom}{Hom}
\DeclareMathOperator{\Frob}{Frob}









% \usepackage{mathtools}
% \usepackage{amssymb}
% \usepackage{amsthm}
% \usepackage{amsmath}
% \usepackage{graphicx}
% \usepackage{xcolor}
% \usepackage{enumerate}
% \usepackage[hidelinks, colorlinks=false]{hyperref}
% % \usepackage{showlabels} % not compatible with Plastex
% \usepackage[normalem]{ulem}

% \newcommand{\rs}[1]{{\color{blue}  RS: #1.}}
% \newcommand{\lars}[1]{{\color{red}  LB: #1.}}
% \newcommand{\asgar}[1]{{\color{green} AJ: #1}} %\color{TealBlue}
% \newcommand{\ct}[1]{{\color{purple}  CT: #1}}


% \theoremstyle{plain}
% \newtheorem{theorem}{Theorem}[section]
% \newtheorem{lemma}[theorem]{Lemma}
% \newtheorem{proposition}[theorem]{Proposition}
% \newtheorem{cor}[theorem]{Corollary}{corollary}
% \theoremstyle{definition}
% \newtheorem{definition}[theorem]{Definition}
% \newtheorem{remark}[theorem]{Remark}
% \newtheorem{example}[theorem]{Example}
% \newtheorem{examples}[theorem]{Examples}
% \numberwithin{section}{chapter}
% \numberwithin{subsection}{section}
% \numberwithin{subsubsection}{subsection}
% \numberwithin{equation}{section}

% \newcommand{\N}{{\mathbb{N}}}
% \newcommand{\Z}{{\mathbb{Z}}}
% \newcommand{\Q}{{\mathbb{Q}}}
% \newcommand{\R}{{\mathbb{R}}}

% \DeclareMathOperator{\ch}{\operatorname{ch}}
% \DeclareMathOperator{\dens}{\operatorname{dens}}
% \DeclareMathOperator{\supp}{\operatorname{supp}}
% \DeclareMathOperator{\tp}{\operatorname{top}}
% \DeclareMathOperator{\im}{\operatorname{im}}
% \DeclareMathOperator{\Lip}{\operatorname{Lip}}
% \DeclareMathOperator{\bd}{\operatorname{bd}}
% \DeclareMathOperator*{\esssup}{\operatorname{ess\,sup}}

% \newcommand{\fp}{{\mathfrak p}}
% \newcommand{\fP}{{\mathfrak P}}
% \newcommand{\fu}{{\mathfrak u}}
% \newcommand{\fU}{{\mathfrak U}}
% \newcommand{\fv}{{\mathfrak v}}
% \newcommand{\fq}{{\mathfrak q}}
% \newcommand{\fQ}{{\mathfrak Q}}
% \newcommand{\fT}{{\mathfrak T}}
% \newcommand{\fL}{{\mathfrak L}}
% \newcommand{\fC}{{\mathfrak C}}
% \newcommand{\pc}{{\mathrm{c}}}
% \newcommand{\ps}{{\mathrm{s}}}
% \newcommand{\AD}{{\bf s}}
% \newcommand{\fc}{{\Omega}}
% \newcommand{\borel}{{\mathcal{E}}}
% \newcommand{\borelb}{{\mathcal{G}}}
% \newcommand{\scI}{{\mathcal{I}}} % note: renamed because it conflicted with another command
% \newcommand{\tQ}{{Q}}
% \newcommand{\mfa}{{\vartheta}}
% \newcommand{\mfb}{{\theta}}
% \newcommand{\Mf}{{\Theta}}
% \newcommand{\fcc}{{\mathcal{Q}}}
